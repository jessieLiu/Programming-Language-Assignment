\documentclass{article}

\usepackage[a4paper,margin=1in]{geometry}
\usepackage{fancyhdr}
\usepackage{amsmath}
\usepackage{booktabs}
\usepackage{listings}
\usepackage{graphicx}
\usepackage{asymptote}

\pagestyle{fancy}
\fancyhf{}
\lhead{Programming Language: Assignment 4}
\rhead{Sui Qingyu (5090309011)}
\cfoot{\thepage}

\setlength\parskip{0.5em}

\lstset{numbers=left,
        numberstyle=\scriptsize,
        frame=lines,
        flexiblecolumns=false,
        language=C,
        basicstyle=\ttfamily\small,
        breaklines=true,
        extendedchars=true,
        escapechar=\%,
        texcl=true,
        showstringspaces=false,
        keywordstyle=\bfseries,
        tabsize=4}

\begin{document}
\section*{Problem 1}
\textbf{Q:} \textit{For the following Clite program, (a) What is the value of the parameter n in topmost stack activation record each time the function Fibonacci calls? (b) How many stack activation records are activated for the call Fibonacci(13)?}

\begin{lstlisting}
int Fibonacci (int n) {
    if (n < 2) return n;
    else return Fibonacci(n-1) + Fibonacci(n-2);
}
int main () {
    int answer;
    answer = Fibonacci(8);
}
\end{lstlisting}

(a) Here are the values of n in turn: 8, 7, 6, 5, 4, 3, 2, 1, 0, 1, 2, 1, 0, 3, 2, 1, 0, 1, 4, 3, 2, 1, 0, 1, 2, 1, 0, 5, 4, 3, 2, 1, 0, 1, 2, 1, 0, 3, 2, 1, 0, 1, 6, 5, 4, 3, 2, 1, 0, 1, 2, 1, 0, 3, 2, 1, 0, 3, 2, 1, 0, 1, 4, 3, 2, 1, 0, 1, 2, 1, 0.

(b) 753 statck activation records are activated.

\section*{Problem 2}
\textbf{Q:} \textit{The following C/C++ program solves the Towers of Hanoi problem for three disks. Draw the Run-Time Stack after each call and return of MoveTower function.}

\begin{lstlisting}
void MoveTower(int disks, char start, char end, char temp)
{
    if(disks ==1)
        cout<<"Move a disk from "<<start<<" to "<<end<<endl;
    else{
        MoveTower(disks-1,start,temp,end);
        cout<<"Move a disk from "<<start<<" to "<<end<<endl;
        MoveTower(disks-1,temp,end,start);
    }
}
int main(int argc, char* argv[]) {
    int totalDisks = 3;
    MoveTower(totalDisks,'A', 'B', 'C');
    return 0;}
\end{lstlisting}

The answer is on the appendix paper.

\section*{Problem 4}
\textbf{Q:} \textit{Run the following Clite program, and draw the Abstract Syntax Sketch (like Figure 10.4) for this program.}

\begin{lstlisting}
int rem (int x, int y){
    return x - x/y * y;
}
int gcd (int x, int y){
    int z;
    if (y == 0) return x;
    else if (x == 0) return y;
    else {
        z = rem(x, y);
        return gcd(y, z);}}
int main () {
    int answer;
    answer = gcd(24,10);}
\end{lstlisting}

The answer is on the appendix paper.

\section*{Problem 5}
\textbf{Q:} \textit{Write down type map $tm_G$, $tm_F$ and $tm_f$ for the following program, where f can be either main, A, B or C.}

\begin{lstlisting}
int h, i, j, k;
void C(int m, int l, int n){
    h = m + l + n;}
void B(int w) {
    int j, k;
    i = 2*w;
    w = w + 1;
    C(i, j, w);}
void A(int x, int y) {
    bool i, j;
    B(h);}
int main () {
    int a, b;
    h = 5; a = 3; b = 2; k = 12;
    A(a, b);}
\end{lstlisting}

\begin{align*}
tm_G = &\{\langle h, int\rangle , \langle i, int\rangle , \langle j, int\rangle , \langle k, int\rangle \} \\
tm_F = &\{\langle C, void, \{\langle m, int\rangle , \langle l, int\rangle , \langle n, int\rangle \}\rangle , \langle B, void, \{\langle w, int\rangle \}\rangle , \\
       &\langle A, void, \{\langle x, int\rangle , \langle y, int\rangle \}\rangle \} \\
tm_C = &(tm_G \cup tm_F) \overline{\cup} \{\langle m, int\rangle , \langle l, int\rangle , \langle n, int\rangle \} \\
     = &\{\langle h, int\rangle , \langle i, int\rangle , \langle j, int\rangle , \langle k, int\rangle , \langle m, int\rangle , \langle l, int\rangle , \langle n, int\rangle , \\
       &\langle C, void, \{\langle m, int\rangle , \langle l, int\rangle , \langle n, int\rangle \}\rangle , \\
       &\langle B, void, \{\langle w, int\rangle \}\rangle , \\
       &\langle A, void, \{\langle x, int\rangle , \langle y, int\rangle \}\rangle \} \\
tm_B = &(tm_G \cup tm_F) \overline{\cup} \{\langle w, int\rangle \} \\
     = &\{\langle h, int\rangle , \langle i, int\rangle , \langle j, int\rangle , \langle k, int\rangle , \langle w, int\rangle , \\
       &\langle C, void, \{\langle m, int\rangle , \langle l, int\rangle , \langle n, int\rangle \}\rangle , \\
       &\langle B, void, \{\langle w, int\rangle \}\rangle , \\
       &\langle A, void, \{\langle x, int\rangle , \langle y, int\rangle \}\rangle \} \\
tm_A = &(tm_G \cup tm_F) \overline{\cup} \{\langle x, int\rangle , \langle y, int\rangle \} \\
     = &\{\langle h, int\rangle , \langle i, int\rangle , \langle j, int\rangle , \langle k, int\rangle , \langle x, int\rangle , \langle y, int\rangle , \\
       &\langle C, void, \{\langle m, int\rangle , \langle l, int\rangle , \langle n, int\rangle \}\rangle , \\
       &\langle B, void, \{\langle w, int\rangle \}\rangle , \\
       &\langle A, void, \{\langle x, int\rangle , \langle y, int\rangle \}\rangle \} \\
\end{align*}

\section*{Problem 6}
\textbf{Q:} \textit{Using the example Clite functions defined in Figures 10.1 and 10.5, construct three different calls; each call should violate one of the Type Rules 10.6, 10.7, and 10.8, but not the other two.}

\begin{lstlisting}
int answer;
string wrong_answer;

\\ This violate Type Rule 10.6
fibonacci(8);

\\ This violate Type Rule 10.7
answer = fibonacci(8, 9, 10);

\\ This violate Type Rule 10.8
wrong_answer = fibonacci(8);
\end{lstlisting}

\section*{Problem 7}
\textit{Oh I think it a little hard for me...}

\section*{Problem 8}
\textbf{Q:} \textit{Read the following Perl program, get it to run, and explain why some of addresses printed are identical.}

\begin{lstlisting}
#!/usr/local/bin/perl
Use strict;
Use warnings;
my @array;
for(0..9){
    my $tmp = 123;
    my $addr = \$tmp;
    Print “$_ has address: $addr\n”;
    $array[$_/2] = $addr;
}
\end{lstlisting}

Because the value of ``\$\_/2'' is 0, 0, 1, 1, 2, 2..., line 9 will not apply for new memory in one of two times. So in these time (2nd, 4th, 6th, 8th and 10th) of the loop, the value of \$addr will not change. Otherwise, the value of \$addr will change because of the memory for it before has been used.
\end{document}
