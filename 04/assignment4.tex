\documentclass{article}

\usepackage[a4paper,margin=1in]{geometry}
\usepackage{fancyhdr}
\usepackage{amsmath}
\usepackage{booktabs}
\usepackage{listings}
\usepackage{graphicx}
\usepackage{asymptote}

\pagestyle{fancy}
\fancyhf{}
\lhead{Programming Language: Assignment 4}
\rhead{Sui Qingyu (5090309011)}
\cfoot{\thepage}

\setlength\parskip{0.5em}

\lstset{numbers=left,
        numberstyle=\scriptsize,
        frame=lines,
        flexiblecolumns=false,
        language=C,
        basicstyle=\ttfamily\small,
        breaklines=true,
        extendedchars=true,
        escapechar=\%,
        texcl=true,
        showstringspaces=true,
        keywordstyle=\bfseries,
        tabsize=4}

\begin{document}
\section*{Problem 1}
\textbf{Q:} \textit{Consider the following sequence of C/C++ statements. Which of them are syntactically valid but have no reasonable semantic interpretation (assuming the i and j have been declared as int variables)? And explain why.}

\begin{lstlisting}
j = 0;
i = 3 / j;
for (i = 1; i > -1; i++)
    i--;
int func(int num) {
    if (num == 0) return 1;
    else return func(num - 1); }
\end{lstlisting}

In line 2, it doesn't have a reasonable interpretation, because integer cannot be divided by 0 in semantic.

In line 4 and 5, there is an infinite loop.

\section*{Problem 3}
\textbf{Q:} \textit{Define, in any programming languages, a function f, such that the evaluation of the expression (a + f(b)) * (C + (f(b))) when performed from left-to-right has a result differs from that obtained by evaluation right-to-left.}
\end{document}
