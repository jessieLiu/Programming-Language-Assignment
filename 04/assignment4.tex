\documentclass{article}

\usepackage[a4paper,margin=1in]{geometry}
\usepackage{fancyhdr}
\usepackage{amsmath}
\usepackage{booktabs}
\usepackage{listings}
\usepackage{graphicx}
\usepackage{asymptote}

\pagestyle{fancy}
\fancyhf{}
\lhead{Programming Language: Assignment 4}
\rhead{Sui Qingyu (5090309011)}
\cfoot{\thepage}

\setlength\parskip{0.5em}

\lstset{numbers=left,
        numberstyle=\scriptsize,
        frame=lines,
        flexiblecolumns=false,
        language=C,
        basicstyle=\ttfamily\small,
        breaklines=true,
        extendedchars=true,
        escapechar=\%,
        texcl=true,
        showstringspaces=true,
        keywordstyle=\bfseries,
        tabsize=4}

\begin{document}
\section*{Problem 1}
\textbf{Q:} \textit{Consider the following sequence of C/C++ statements. Which of them are syntactically valid but have no reasonable semantic interpretation (assuming the i and j have been declared as int variables)? And explain why.}

\begin{lstlisting}
j = 0;
i = 3 / j;
for (i = 1; i > -1; i++)
    i--;
int func(int num) {
    if (num == 0) return 1;
    else return func(num - 1); }
\end{lstlisting}

In line 2, it doesn't have a reasonable interpretation, because integer cannot be divided by 0 in semantic.

In line 3 and 4, there is an infinite loop. There's not a final state, as variable \textit{i} flips between 0 and 1.

In line 5 to 7, there will be an infinite loop if variable \textit{num} is less than 0.

\section*{Problem 3}
\textbf{Q:} \textit{Define, in any programming languages, a function f, such that the evaluation of the expression (a + f(b)) * (c + (f(b))) when performed from left-to-right has a result differs from that obtained by evaluation right-to-left.}

In most language supports ``++'' operator, like C and C++, f(b) = b++ is an example that l2r differs from r2l evaluation.

\section*{Problem 4}
\textbf{Q:} \textit{Consider the expression x + y / 2 in the language C. How many different meanings does this expression have, depending on the types of x and y?}

When y is an integer (int, short, etc.), this sentence means x plus the integer part of the result x divided by y.

When y is an float number (float, double, etc.), it means the exact result of equation ``x + y / 2''.

\section*{Problem 6}
\textbf{Q:} \textit{Using Table 7.3 (on text book, page 162) as a guide, write a trace table for the Clite program:}

\begin{lstlisting}
void main() {
    int i, a, z;
    i = 5;
    a = 2;
    z = 1;
    while(i>0) {
        if(i –i/2*2 == 1)
            z = z * a;
        i = i /2;
        a = a * a; } }
\end{lstlisting}

The answer is shown in Table \ref{t:tracetable}.

\begin{table}[htbp]
\centering
\begin{tabular}{ccccc}
\toprule
 & Before & \multicolumn{3}{c}{Variables} \\
\cline{3-5}
Step & Statement & i & a & z \\
\midrule
1 & 3 & undef & undef & undef \\
2 & 4 & 5 & undef & undef \\
3 & 5 & 5 & 2 & undef \\
4 & 6 & 5 & 2 & 1 \\
5 & 7 & 5 & 2 & 1 \\
6 & 8 & 5 & 2 & 1 \\
7 & 9 & 5 & 2 & 2 \\
8 & 10 & 2 & 2 & 2 \\
9 & 6 & 2 & 4 & 2 \\
10 & 7 & 2 & 4 & 2 \\
11 & 9 & 2 & 4 & 2 \\
12 & 10 & 1 & 4 & 2 \\
13 & 6 & 1 & 16 & 2 \\
14 & 7 & 1 & 16 & 2 \\
15 & 8 & 1 & 16 & 2 \\
16 & 9 & 1 & 16 & 32 \\
17 & 10 & 0 & 16 & 32 \\
18 & 6 & 0 & 256 & 32 \\
19 & 10 & 0 & 256 & 32 \\
\bottomrule
\end{tabular}
\caption{Trace table for the Clite program in Problem 6}
\label{t:tracetable}
\end{table}

\section*{Problem 7}
\textbf{Q:} \textit{Suppose $state_1 = \{<x, 1>, <y, 2>, <z, 3>\}$, $state_2 = \{<y, 5>\}$, and $state_3 = \{<w, 1>\}$. What are the results of the following operations?}

\textbf{a.} $state_1\ \overline{\cup}\ state_2 = \{<x, 1>, <y, 5>, <z, 3>\}$

\textbf{b.} $state_1\ \overline{\cup}\ state_3 = \{<x, 1>, <y, 2>, <z, 3>, <w, 1>\}$

\textbf{c.} $\O\ \overline{\cup}\ state_2 = \{<y, 5>\}$

\textbf{d.} $state_2\ \otimes\ state_3 = \O$

\textbf{e.} $(state_1 - (state_1\ \otimes\ state_3))\ \overline{\cup}\ state_3 = \{<x, 1>, <y, 2>, <z, 3>, <w, 1>\}$

\section*{Problem 8}
\textbf{Q:} \textit{Show all steps in the derivations of the meaning of the following assignment statement when execute in the given state, using the semantic rules given in Section 8.2.3: $M(z=2*x+3/y-4,\{<x,6>,<y,-12>,<z,75>\})$}

\begin{align*}
M(2) &= integer 2 \\
M(x) &= integer 6 \\
M(3) &= integer 3 \\
M(y) &= integer -12 \\
M(4) &= integer 4 \\
M(2 * x) &= (integer 2) int* (integer 6) = integer 12 \\
M(3 / y) &= (integer 3) int/ (integer -12) = integer 0 \\
M(2 * x + 3 / y) &= (integer 12) int+ (integer 0) = integer 11 \\
M(2 * x + 3 / y - 4) &= (integer 11) - (integer 4) = integer 7 \\
M(z = 2 * x + 3 / y - 4) &= state \{<x,6>,<y,-12>,<z,7>\}
\end{align*}

\end{document}
